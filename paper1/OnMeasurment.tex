\documentclass[11pt]{article}
\usepackage{amsmath, amsthm, amssymb, physics, geometry, mdframed, authblk}
\usepackage{xcolor}
\usepackage{microtype}
\usepackage{booktabs}
\usepackage{thmtools}
\usepackage[hidelinks]{hyperref}
\usepackage[nameinlink]{cleveref}
\usepackage{mathrsfs}
\geometry{margin=1in}

\title{On Measurement: The Relativity of Information Frames}
\author[ ]{Gabriel Masarin}
\affil[ ]{\texttt{g.masarin@proton.me}}
\date{}
\theoremstyle{plain}
\newtheorem{theorem}{Theorem}[section]
\newtheorem{lemma}[theorem]{Lemma}
\newtheorem{corollary}[theorem]{Corollary}
\newtheorem{proposition}[theorem]{Proposition}
\newtheorem{remark}[theorem]{Remark}
\usepackage{setspace}
\setstretch{1.07}
\clubpenalty=10000
\widowpenalty=10000
\setlength{\belowdisplayskip}{0.8\baselineskip}
\setlength{\abovedisplayskip}{0.8\baselineskip}
\theoremstyle{definition}
\newtheorem{definition}[theorem]{Definition}
\newmdenv[
  backgroundcolor=gray!10,
  linecolor=gray!50,
  linewidth=1pt,
  roundcorner=6pt,
  innertopmargin=8pt,
  innerbottommargin=8pt,
]{principlebox}
\theoremstyle{remark}
\newtheoremstyle{axiomstyle}% <name>
  {8pt}% <Space above>
  {8pt}% <Space below>
  {\itshape}% <Body font>
  {}% <Indent amount>
  {\bfseries}% <Theorem head font>
  {.}% <Punctuation after theorem head>
  {0.5em}% <Space after theorem head>
  {\thmname{#1}\thmnumber{ #2}\thmnote{ (#3)}}% <Theorem head spec>

\theoremstyle{axiomstyle}
\newtheorem{axiom}{Axiom}
\begin{document}
\maketitle

\vspace{0.5em}
\begin{abstract}
    We introduce a structural framework based on a symmetry principle we call the
    \emph{Relativity of Information Frames} (RIF). The framework treats information
    frames as fundamental objects and constrains how their event structures may be
    jointly realized during interaction. Using measure theory and contextuality, we
    formalize interactions as embeddings into a joint measurable space and show that
    enforcing perspectival symmetry induces a canonical coarse-graining of events. From
    a single RIF axiom, we derive a distinguished $\sigma$-algebra - the 
    \emph{pointer algebra} - and prove that it is the maximally informative algebra
    admitting non-contextual probability measures. Any consistent empirical family admits
    a global probability measure on the pointer algebra, whereas any strict extension
    generically reintroduces contextuality.

    We then apply the framework to quantum theory. Without attempting a full reconstruction
    we show how standard quantum structure are naturally represented: observables arise as
    perspectival maps, probability emerges from coarse graining, and the born rule follows
    from a martingale argument. Wigner-type scenarios are shown to be consistent via the
    tower law, and unitary evolution is intrepreted as a manifestation of gauge freedom
    in the choice of embeddings. We further discuss how the induced coarse-graining
    introduces an intrinsic arrow of time and outline possible connections to causal 
    structure and geometry. The results are structural rather than dynamical and provide
    a unified perspective on contextuality, probability, and measurement.
\end{abstract}

\section{Introduction}
This work develops a structural framework motivated by a simple but persistent tension in
the foundations of physics: the apparent incompatibility between globally consistent
descriptions of physical systems and the perspectival nature of information obtained through
interaction. While the framework did not initially arise from a single ghiding principle,
its development ultimately crystallized around a symmetry statement, which we call the
\textbf{Relativity of Information Frames} (RIF).

At its core, RIF asserts that no informational perspective should be privileged over another 
in the description of physical events. This principle is not imposed dynamically, but
structurally: it constrains which event structures can be jointly realized when multiple
information frames interact. The framework developed here formalizes this idea using
measure theory and contextuality, treating information frames as the fundamental ontic
objects and interactions as operations that enforce perspectival consistency.

Part~I of this paper establishes the formal framework. We introduce information frames,
their embeddings into joint measurable spaces, and a notion of structural contextuality
that captures when full informational structures cannot be jointly preserved. From a
single RIF axiom, we derive a canonical coarse-graining of the joint event structure and
prove that the resulting $\sigma$-algebra is, in a precise sense, the maximally informative
non-contextual algebra compatible with the interaction. We refer to this algebra as the
\textbf{pointer algebra}. We further show that, while probability measures need not exist
on the full joint structure, any consistent empirical family admits a global probability
measure on the pointer algebra, and that any strict extension of this algebra reintroduces
contextuality.

Part~II applies the framework to quantum theory. Without attempting a full reconstruction,
we show how standard quantum structures are naturally represented within RIF. Observables
arise as perspectival maps, conditioning corresponds to interaction, and probability
assingnments emerge from coarse-graining rather than being postulated. We derive the
Born rule operationally via a martingale argument, establish the consistency of Wigner-type
scenarios using the tower law, and interpret unitary evolution as a manifestation of
gauge freedom in the choice of embeddings. We also outline how familiar gauge symmetries
may be viewed as relations between information frames.

Beyond quantum mechanics, we explore several conceptual consequences of the framework. That
enforced coarse-graining introduces an intrinsic irreversibility, providing a structural 
arrow of time. Interactions define a relational graph of frames that suggests a notion
of causal locality, and we briefly discuss how the degree of coarse-graining may be related
to geometric and gravitational considerations.

The results presented here are structural rather than dynamical. No specific dynamics are 
assumed, and no new empirical predictions are claimed. Instead, the framework offers a
unified perspective on contextuality, probability, and measuremen, clarifying the origin
of quantum features that are often treated as axiomatic. We conclude by comparing
RIF to existing approaches, including decoherence-based accounts, and by outlining 
directions for further development.
\section{Motivation - The Relativity of Information Frames}

Consider two physical observers, Alice and Bob, each equipped with a clock and a ruler.
To infer a particle’s momentum, they make two position measurements and record 
the elapsed time.

However,
\begin{itemize}
    \item If they agree on the spatial separation, they must disagree on the elapsed time;
    \item If they agree on the elapsed time, the measured spatial separation must differ.
\end{itemize}

Their interactions with the world differ — and so does what each can resolve as an event.

What Alice calls “particle at position $x$ at time $t$” is determined by her 
interaction channels and detection thresholds.

Thus there is no global, frame-independent $\sigma$-algebra of events.  
Every physical system carries its own information frame: a $\sigma$-algebra 
of distinguishable outcomes accessible through its interactions.

Einstein taught that coordinate descriptions are relative while causal order is invariant.  
We extend this principle.

\begin{principlebox}
\begin{center}
\textbf{Relativity of Information Frames (RIF)}

Nature does not privilege one information frame over another. What is physical is what
all information frames can agree upon.
\end{center}

\end{principlebox}

Measurement is not the revelation of a pre-existing global state; it is the joint refinement
(and, when necessary, coarse-graining) of information frames when systems interact.  
From this symmetry, quantum state update, pointer bases, and even causal 
geometry follow as consequences.

\part{Structural Framework}
\section{Background}
\subsection{Measure Theory}
A full account of measure theory and probability theory is outside the scope of this paper;
for a standard reference, see \cite{Williams1991}. In this section we introduce only 
the notion of a probability space and the measure-theoretic concepts that will be 
explicitly used later. The exposition is intentionally brief and self-contained, though 
some familiarity with the subject is helpful.

\subsubsection*{Probability Spaces}
In measure-theoretic probability, a probability space is given by a triple:
\[
    (\Omega, \mathcal F, \mu)
\]
where each component encodes a distinct element of a probality model.
\subsubsection*{The Sample Space $\Omega$}
The sample space $\Omega$ is the set of all possible outcomes of an experiment. It's elements
$\omega \in \Omega$ represent individual realizations or trials. In general, $\Omega$ is
endowed with additional structure beyond serving as the underlying space from which outcomes
are drawn.

Depending on the context, elements of $\Omega$ may correspond to coin toss outcomes, 
experiment runs, or realizations of an abstract physical system.
\subsubsection*{The $\sigma$-algebra $\mathcal F$}
The $\sigma$-algebra $\mathcal F$ specifies which subsets of $\Omega$ are considered 
\textbf{events}. Events are thus sets of outcomes $\omega \in \Omega$ to which probabilities
may be assigned.

\begin{definition}[$\sigma$-algebra]\label{def:sigma}
    A $\sigma$-algebra $\mathcal F$ is a collection of subsets of $\Omega$ such that:
    \begin{enumerate}
        \item $\empty,\Omega \in \mathcal F$
        \item Closed under complemetns. $F \in \mathcal F \rightarrow F^{c} \in \mathcal F$
        \item Closed under countable unions. 

            \[\left\{F_i\right\}_{i \in \mathbb N} \in \mathcal F \rightarrow 
            \bigcup_{i \in \mathbb N} F_i \in \mathcal F\]
    \end{enumerate}
\end{definition}
The $\sigma$-algebra encodes the events that are meaningfully distinguishable within the
model and therefore forms the central structural component of a probability space.
\subsubsection*{The Probability Measure $\mu$}
A probability measure assigns probabilities to events in $\mathcal F$.
\begin{definition}[Probability Measures]\label{def:prob}
    A probability measure is a function $\mu: \mathcal F \to [0,1]$ satisfying: 
    \begin{enumerate}
        \item Total probabilities:
            \[
                \mu(\emptyset) = 0 \quad \mu(\Omega) = 1
            \]
        \item For any countable collection of pairwise disjoint sets 
            $\left\{F_i\right\}_{i \in \mathbb N}$,
            with $F_i \cap F_j = \emptyset$ for all $i \neq j$.
            \[
                \mu\left(\bigcup_{i \in \mathbb N} F_i\right) = 
                \sum_{i \in \mathbb N} \mu(F_i)
            \]
    \end{enumerate}
In this work, probability measures will be intrepreted as states on a measurable space. 
Encoding how probabilities are distributed over the events in $\mathcal F$.
\end{definition}
\subsubsection*{Measurable Functions}\label{def:mes-fun}
A \textbf{measurable function}, often refered to in probability theory as a 
\textbf{random variable}, is a function between measurable spaces 
\[
f: \Omega_1 \to \Omega_2
\]
such that:
\[\
    \forall A \in \mathcal F_2\; \quad\; f^{-1}(A) \in \mathcal F_1
\]
That is, the preimage of every event in $\mathcal F_2$ is a event in $\mathcal F_1$.
\subsubsection*{Pushfoward Measure}
Given a measurable function 
\[
    f: \Omega_1 \to \Omega_2
\] 
And a probability measure $\mu$ in $\left(\Omega_1, \mathcal F_1\right)$, we can define 
a probability measure on $\left(\Omega_2,\mathcal F_2\right)$ by
\[
    \left(f_{*} \mu\right)(A) := \mu\!\left(f^{-1}(A)\right), 
    \qquad A \in \mathcal F_2
\]
The measure $f_* \mu$ is called the \textbf{pushfoward} of $\mu$ along $f$.

The pushfoward measure represents the probability distribution induced on $\Omega_2$ by
the map $f$ when the underlying space is described by the measure $\mu$.
\subsubsection*{Embeddings}
We will be particulary interested in certain classes of measurable maps that preserve the
structure of measurable spaces.
\begin{definition}[Measurable Isomorphism]\label{def:mes-iso}
    A measurable function $T: \Omega_1 \to \Omega_2$ is called a 
    \textbf{measurable isomorphism} if it is \textbf{bijective} and it's inverse 
    $T^{-1}: \Omega_2 \to \Omega_1$ is also measurable. In this case, the measurable 
    spaces are \textbf{isomorphic}, denoted:
    \[
        \left(\Omega_1, \mathcal F_1\right) \cong \bigl(\Omega_2, \mathcal F_2\bigr)
    \]
    \noindent
    Such maps preserve the full measurable structure.
\end{definition}
\begin{definition}[Measurable Space Automorphisms]\label{def:mes-auto}
    For a measurable space $\left(\Omega,\mathcal F\right)$, the group of measurable 
    automorphisms is
\[
    \mathrm{Aut}\left(\Omega,\mathcal F\right) := \left\{T: \Omega \to \Omega \;\mid\; T
    \text{ is bijective, } T\text{ and }T^{-1}\text{ are measurable.}\right\}
\]
\end{definition}
While measurable isomorphisms preserve the entire structure of a space, our work requires
maps that allow a measurable space to be faithfully represented within a larger one.
\begin{definition}[Measurable Embedding]\label{def:mes-emb}
    A \textbf{measurable embedding}  is a injective measurable function
    \[
        \iota: \Omega_1 \to \Omega_2
    \]
    such that the inverse map
    \[
        \iota^{-1}: \iota(\Omega_1) \to \Omega_1
    \]
    is measurable, where $\iota(\Omega_1)$ is equipped with the sub-$\sigma$-algebra 
    induced from $\mathcal F_2$.
\end{definition}
In this case, $\iota$ identifies $(\Omega_1, \mathcal F_1)$ with a measurable subspace of
$(\Omega_2, \mathcal F_2)$, preserving the $\sigma$-algebra of the source space.
\begin{definition}[Embedding-induced $\sigma$-algebra]\label{def:ind-emb-alg}
Let $\iota:\Omega_1\to\Omega_2$ be a measurable embedding.  
The \emph{embedding-induced $\sigma$-algebra} of $\iota$ is the sub-$\sigma$-algebra on
$\iota(\Omega_1)\subset\Omega_2$ induced from $\mathcal F_2$, defined by
\[
\mathcal F^{\iota}
\;:=\;
\mathcal F_2\!\restriction_{\iota(\Omega_1)}
\;=\;
\{\,A\cap\iota(\Omega_1)\mid A\in\mathcal F_2\,\}.
\]
\noindent
The same sub-$\sigma$-algebra used in the definition above.
\end{definition}
\subsection{Contextuality}
A central concept underlying the theory developed in this work is that of 
\textbf{contextuality}. The definitions presented here are adapted from the 
sheaf-theoretic formulation of contextuality introduced in \cite{Abramsky&Brandenburger},
to a measure-theoretic framework.
\subsubsection*{Labels and Contexts}
We begin by introducing the notion of a measurement label. Intuitively, a measurement 
label represents a physical distinction that can be probed in a system. 
Measurement labels encode the primitive degrees of freedom of the model.
\subsubsection*{Measurement Labels}
\begin{definition}[Measurement labels]\label{def:mes-label}
    A \textbf{measurement label} is an abstract symbol $m$ associated with an 
    measurable space $(\Omega_m, \mathcal F_m)$, representing the possible outcomes of 
    measuring $m$. We write
    \[
        m \longmapsto (\Omega_m, \mathcal F_m)
    \]
    The collection of all measurement labels cosidered by a model considers is denoted
    $\mathcal M$.
\end{definition}
\subsubsection*{Global Space}
\begin{definition}[Global measurable space]\label{def:global-space}
    The \textbf{global space}, representing the joint space of all measurement labels, is
    the product measurable space
    \[
        \left(\Omega_{\mathcal M}, \mathcal F_{\mathcal M}\right) := 
        \left(\prod_{m \in \mathcal M} \Omega_m \;, \bigotimes_{m \in \mathcal M} 
        \mathcal F_m \right)
    \]
    \noindent
\end{definition}
Here $\bigotimes_{m \in \mathcal M} \mathcal F_m$ denotes the product $\sigma$-algebra
generated by cylinder sets. For two measurable spaces $(\Omega_1,\mathcal F_1)$ and 
$(\Omega_2, \mathcal F_2)$ the product $\sigma$-algebra is given by
\[
\mathcal F_1 \otimes \mathcal F_2 := 
    \sigma\left(\left\{F_1 \times F_2: F_1 \in \mathcal F_1, F_2 \in 
    \mathcal F_2\right\}\right)
\]
We emphasize that no probability measure is specified on the global space at this stage.
Our interest here lies in the measurable structure itself, independently of any particular
choice of global state.
\subsubsection*{Contexts}
We now introduce the notion of a context. A context represents a collection of measurement
labels that are jointly accessible to an observer, or equivalently, a set of degrees of
freedom that can be meaningfully considered together. Intuitively, a context specifies
the degrees an observer may simultaneously interact with on the system.
\begin{definition}[Context]\label{def:context}
    A \textbf{context} is a finite subset $C \subseteq \mathcal M$ of measurement labels.
    To each context we associate a measurable space
    \[
        (\Omega_{C}, \mathcal F_{C}) := \left(\prod_{m \in C}
        \Omega_m \;, \bigotimes_{m \in C} \mathcal F_m \right)
    \]
\end{definition}
\subsubsection*{Canonical Context Projections}
Within a context, we define canonical projection maps onto the outcome spaces of 
individual measurament labels.
\begin{definition}[Canonical context projections]\label{def:can-proj}
For a context $C$ and a label $m \in C$, the canonical projection is the measurable
function
\[
    \pi_{C \to \{m\}}: \Omega_C \to \Omega_m
\]
defined by
\[
    \pi_{C \to \{m\}}\left(\omega\right) = \omega_m \qquad \forall \omega \in \Omega_C
\]
More generally, for any subcontext $D \subseteq C$, we define the projection
\[
    \pi_{C \to D}: \Omega_C \to \Omega_D
\]
by restriction to the coordinates indexed by $D$.
\end{definition}
Intuitively, the projection $\pi_{C \to \{m\}}$ the \textbf{perspective} $C$ has on the
measurament label $m$. The inverse images of projections define canonical measurable
subsets of a context space.
\begin{definition}[Cylinder Sets]\label{def:cyl-sets}
    Let $C$ be a context and $m \in C$. For any $A \in \mathcal F_m$, the corresponding
    \textbf{cylinder set} in $\Omega_C$ is defined by
    \[
        \pi_{C\to\{m\}}^{-1}(A) := \left\{\omega \in \Omega_C \;\mid\; \omega_m \in A\right\}
    \]
    \noindent
    Since $\pi$ is a measurable function, all cylinder sets belong to $\mathcal F_C$.
\end{definition}
Contexts do not introduce independent information: all events in a context arise as 
pullbacks of events associated with its measurement labels via the canonical projections.
In this sense, a context contains no information that does not originate from its 
constituent labels.
\subsubsection*{Empirical Model}
We now introduce the first concept that explicitly involves probability measures: the
notion of an \textbf{empirical model}.

Intuitively, an empirical model represents a particular realization of the system as
accessed through different contexts. Equivalently, it may be viewed as a family of
probability distributions describing the observable statistics associated with each
context, subject to consistency on overlaps.
\begin{definition}[Empirical model]\label{def:emp-model}
    An \textbf{empirical model} is a family
    \[
        \{\mu_C\,\}_{C \in \mathcal C}
    \]
    of probability measures, where for each context $C$,
    \[
        \mu_C \text{ is a probability measure on } (\Omega_C, \mathcal F_C)
    \]
    
    These measures are required to satisfy the following \emph{compatibility condition}:
    For all $C,C' \in \mathcal C$ and all subcontexts $D \subseteq C \cap C'$,
    \[
        \left(\pi_{C \to D}\right)_* \mu_C \; = \; \left(\pi_{C' \to D}\right)_* \mu_{C'}
    \]
    This condition expresses the requirement that the probability distributions assigned to
    different contexts agree on their common measurement labels, and hence represent
    consistent marginals of a single underlying empirical situation.
\end{definition}
\subsubsection*{Contextuality}
Empirical models allow us to define a central notion driving the framework developed in
this work, that of \textbf{contextuality}. Informally, contextuality captures the failure
of different observational perspectives to arise as consistent restrictions of a single
global description.
\begin{definition}[Contextuality]\label{def:prob-contextuality}
    Let $\left\{\mu_{C} \right\}_{ C \in \mathcal C}$ be an empirical model.
    The empirical model is said to be \textbf{contextual} if there exists no probability
    measure $\mu$ on the global measurable space $(\Omega_\mathcal M, \mathcal F_\mathcal M)$
    such that
    \[
        \left(\pi_{\mathcal M \to C}\right)_* \mu \; = \;
        \mu_{C} \qquad \forall C \subset \mathcal C
    \]
    \noindent
    If such a probability measure exists, the empirical model is called 
    \textbf{non-contextual}.
\end{definition}
Intuitively, non-contextuality means that the probabilistic data obtained from all 
contexts can be understood as arising from a single joint probability distribution on the
global space, with each context revealing only a partial perspective of that global state.

Contextuality, by contrast, indicates that no such global probability measure exists: 
although each context admits a well-defined probabilistic description, these descriptions
cannot be combined into a single coherent global model.

The existence of contextual empirical models is not merely a formal possibility.
It is well established that there are experimental scenarios for which no
non-contextual global description exists; a detailed analysis can be found in 
\cite{Abramsky&Brandenburger}. The notion introduced here corresponds to probabilistic
contextuality in the sense of \cite{Abramsky&Brandenburger}, expressed in measure-theoretic
language.
\subsection*{Standing Assumptions}
All measurable spaces considered in this work are assumed to be standard Borel spaces. 
Probability measures are taken to be Borel probability measures satisfying the 
usual regularity conditions required for the constructions used below.
\section{Structural Contextuality}
To better understand RIF, we consider formulations of contextuality that do not rely on
the specification of particular probability distributions. Instead, we focus on the
underlying structural constraints imposed by measurable spaces themselves.

This perspective is inspired by the sheaf-theoretic approach to contextuality, but goes
beyond probabilistic inconsistency. In particular, it aligns with1 the notion of 
\textbf{strong contextuality} as defined in \cite{Abramsky&Brandenburger}, where obstruction
arises already at the level of compatible local descriptions.

\subsection{Information Frames}
We begin by introducing the notion of an \textbf{information frame}.

\begin{definition}[Information Frame]\label{def:info-frame}
    Let $C \subseteq \mathcal M$ be a context. An \textbf{information frame} over $C$ is
    a measurable space of the form
    \[
        \mathscr I_{C, \mathcal F} := \left(\prod_{m \in C} \Omega_m, \mathcal F\right)
    \]
    Where $\mathcal F$ is a $\sigma$-algebra satisfying
    \[
        \mathcal F \; \subseteq\; \bigotimes_{m \in C} \mathcal F_m
    \]
    \noindent
    The $\sigma$-algebra $\mathcal F$ represents the set of distinctions that the frame is
    able to make about the measurement labels in $C$.
\end{definition}

An information frame may be interpreted as a perspective on the system: it specifies what
can, in principle, be distinguished within the context $C$. Probability measures on
$\mathscr I_{C, \mathcal F}$ then represent particular states compatible with that 
perspective.

We note that when a probability measure is specified on an information frame, events are
understood operationally up to null sets. That is, events that differ only on a set of
measure zero are identified as representing the same distinction from the perspective of
that frame.

For notational convenience, when no ambiguity arises we write
\[
    \mathscr I_i \; = \; \mathscr I_{C_i, \mathcal F_i}
\]
\subsection{Structural Contextuality}
\subsubsection*{Shared Events}
First we must establish when two events in different contexts carry \textbf{shared} meaning.
\begin{definition}[Shared Event]\label{def:shared-event}
    Let $C,C' \in \mathcal C$ be contexts with nonempty intersection
    \[
        L := C \cap C' \neq \emptyset
    \]
    An event $E_L \in \mathcal F_L$ is called a \textbf{shared event} of $C$ and $C'$.

    The corresponding events in the context spaces $(\Omega_C, \mathcal F_C)$ and
    $(\Omega_{C'},\mathcal F_{C'})$ are given by the cylinder sets
    \[
        E_C := \pi_{C\to L}^{-1}(E_L), \qquad E_{C'} := \pi_{C'\to L}^{-1}(E_L)
    \]
\end{definition}
\subsubsection*{Embeddings and Event images}
We also note that, since all spaces considered are standard Borel, any embedding used in
this work is understood to be a Borel embedding. In particular, the image of an embedding
is a measurable subset of the target space. Consequently, for a embedding
\[
    \iota: (\Omega_1, \mathcal F_1) \to (\Omega_2,\mathcal F_2)
\]
and any event $E \in \mathcal F_1$, we have
\[
    \iota(E) \in \mathcal F_2
\]
In fact, for the purposes of this work, general $\sigma$-algebra homomorphism suffice.
Under the standing assumptions, any measurable embedding induces an injective 
$\sigma$-algebra homomorphism. We therefore freely move between these equivalent 
perspectives when convenient.
\subsubsection*{Structural Contextuality}
We can now make precise a definition of contextuality that does not rely on probability
measures.
\begin{definition}[Structural Contextuality]\label{def:struc-context}
    Given a family of contexts $\mathcal C$ over $\mathcal M$ with corresponding 
    information frames $\mathscr I_{C, \mathcal F_C}$ and a global space 
    $\left(\Omega, \mathcal F\right)$. 

    The family is said to be \textbf{structuraly contextual} if there exists no family of 
    \textbf{embeddings} into a common sub-$\sigma$-algebra
    $\mathcal G \subseteq \mathcal F$,
    \[
        \iota_{C \in \mathcal C}: \mathscr I_{C, \mathcal F_C} \to 
        \left(\Omega, \mathcal G\right) \qquad C \in \mathcal C
    \]
    Such that for every pair $C,C'$ with $L = C \cap C'$ and every \textbf{shared event}
    $E_L\in \mathcal F_L$,
    \[
        \iota_C(\pi_{C\to L}^{-1}(E_L)) = \iota_{C'}(\pi_{C'\to L}^{-1}(E_L))
    \]
    \noindent
    If such a family of embeddings exists, the family of frames is called 
    \textbf{structurally non-contextual}.
\end{definition}

Intuitvely, structural contextuality relies criticaly on the failure of existence of 
\emph{any} compatible embedding. That is, the events structure of the contexts are 
incompatible to the extent that no faithful realization can identify shared events
globally while preserving the full informational content of each context.
\subsubsection*{Relationship with strong contextuality}
While this notion is distinct from probabilistic contextuality, it is closely related.
When event algebras fail to embed compatibly, any attempt to assign a single global
probability measure respecting all contextual distinctions necessarily requires 
coarse-graining, and may fail entirely.

The existence of \textbf{strong contextuality} in the sense of \cite{Abramsky&Brandenburger} 
guarantees that structural contextuality occurs for suitably fine event structures. In
particular the Kochen--Specker results \cite{Kochen-Specker} show that retaining the full
$\sigma$-algebra structure of measurement contexts eventually obstructs any global 
realization. This highlights the structural inevitability of contextuality in sufficiently
rich logics.

As a illustration in the product space, this can be viewed as the failure of cylinder sets 
to capture the full complexity of the contextual event structures. Consider, for example,
two embeddings
\[
    \iota_1(x) = (x, f(x)) \qquad \iota_2(y) = (g(y), y) 
\]
Here the embedding is fixed on the degrees of freedom controlled by each context, while
the remaining components are unconstrained. Structural contextuality does not arise from a 
lack of freedom in choosing the functions $f$ and $g$, but from the requirement that they
simultaneously preserve the full event structures of the contexts. When shared events are
refined incompatibly across contexts, no choice of $f$ and $g$ can reconcile all induced
events in a single global-algebra.
\section{The Relativity Of Information Frames}
\subsection{Interaction}
We now make precise the meaning of the \textbf{Relativity of Information Frames (RIF)}.
The central objects of the theory are \textbf{Information Frames} from \cref{def:info-frame},
which are taken as the only ontic objects. All physical content arises from their
interactions.

We define interaction structurally, without reference to dynamics or time, as the
co-realization of multiple information frames within a common joint frame.

Before proceeding, we note that when writting
\[
    \iota_C(\mathcal F_C) \subseteq \mathcal G 
\]
we implicitly refer to a sub-$\sigma$-algebra $\mathcal G$ of the codomain induced by the
embedding $\iota_C$. Different choices of embeddings are related by measurable automorphisms
of the codomain and are therefore treated as \textbf{gauge-equivalent}.

\begin{definition}[Joint Frame]\label{def:joint-frame}
    Let $\{\mathscr I_i\}_{i \in I}$ be a family of information frames, with 
    $\mathscr I_i = (\Omega_{C_i}, \mathcal F_i)$.
    Define the joint label set
    \[
        \mathcal C := \bigcup_{i \in I} C_i
    \]
    The \textbf{joint frame} is the measurable space
    \[
        \mathcal J_I := (\Omega_{\mathcal J_I}, \mathcal F_{\mathcal J_I})
    \]
    where
    \[
        \Omega_{\mathcal J_I} := \prod_{m \in \mathcal C} \Omega_m, \qquad
        \mathcal F_{\mathcal J_I} := \sigma\! \left(\bigcup_{i \in I} 
        \iota_i(\mathcal F_i)\right)
    \]
\end{definition}
The joint frame represents the space capable of expressing all distinctions accessible to
the interacting frames. While such frame is always definable at the level of 
measurable structure, structural contextuality may obstruct the existence of a 
$\sigma$-algebra in which all shared events are consistently identified.

Once a joint realization is fixed, the embeddings $\{\iota_i\}$ are not allowed to vary 
within that realization. Different realizations related by automorphisms are treated as
gauge-equivalent, but embeddings are fixed within each gauge choice.
\subsection{Local Perspectives}
To make the relativity principle precise, we require a way to compare contexts within a
fixed joint realization. Although a joint frame $\mathcal J_I$ is constructed via
embeddings $\{\iota_i\}_{i \in I}$, the contextual frames themselves do not directly live
in the joint space.

This comparison is achieved through the following maps.

\begin{definition}[Local Perspective Maps]\label{def:local-per}
    Let $\mathcal J_I = (\Omega_{\mathcal J_I}, \mathcal F_{\mathcal J_I})$ be a joint
    frame generated by the embeddings
    \[
        \iota_i : \Omega_{C_i} \hookrightarrow \Omega_{\mathcal J_I}
    \]
    The \textbf{local perspective map} associated with frame $\mathscr I_i$ is the
    measurable map
    \[
        e_i\; :=\; \iota_i \circ \pi_{\mathcal J_I \to C_i}:\; \Omega_{\mathcal J_I} \to
        \Omega_{\mathcal J_I}
    \]
    \noindent
    \textbf{Note}:
    Once more, under the standing assumptions adopted throughout this work, each local
    perspective map $e_i$ is measurable and therefore induces a endomorphism of the joint
    $\sigma$-algebra via pullback. By abuse of notation, we use the same symbol $e_i$
    to denote both the measurable map on $\Omega_{J_I}$ and its induced action on events.
\end{definition}
The maps $e_i$ represent the full event structure of the information frame $\mathscr I_i$
as embedded in the joint frame, intuitively the perspective of that frame 
on the interaction. 

\begin{proposition}[$e_i$ are idempotent]\label{prop:per-idemp}
    The local perspective maps $e_i$ are idempotent measurable endomorphisms of 
    $(\Omega_{\mathcal J_I}, \mathcal F_{\mathcal J_I})$.
\end{proposition}
\begin{proof}
    By definition $e_i = \; \iota_i \circ \pi_{\mathcal J_I \to C_i}$. Since 
    $\pi_{\mathcal J_I \to C_i} \circ \iota_i = \mathrm{id}_{\Omega_{C_i}}$
    we have $e_i \circ e_i = e_i$
\end{proof}
\begin{remark}
    The maps $e_i$ act as coarse-grainings of the joint event structure: two events are 
    identified whenever they induce the same event on the contextual frame $C_i$.
\end{remark}
\subsection{The Symmetry Of Information}
\subsubsection*{Privilege}
We are now ready to introduce the central concept underlying the \textbf{Relativity of 
Information Frames} the concept of privilege.

\begin{definition}[Privilege]\label{def:privilege}
    We say that a joint frame $\mathcal J_I$ privileges $\mathscr I_i$ over $\mathscr I_j$ 
    for a shared event $E \in \iota_i(\mathcal F_i) \cap \iota_j(\mathcal F_j)$ if, 
    for the corresponding local perspective maps $e_i,e_j$ we have
    \[
        e_j(E) \neq e_j(e_i(E))
    \]
    Equality is understood at the level of events in the joint $\sigma$-algebra
    (or up to null set equivalences under the standing assumptions).
\end{definition}
Intuitively, a joint frame privileges $\mathscr I_i$ over $\mathscr I_j$ whenever 
the perspective of $\mathscr I_i$ alters what $\mathscr I_j$ sees. That is,
conditioning on $\mathscr I_i$'s interpretation of an event changes $\mathscr I_j$'s 
interpretation.

We note that privilege is witnessed by a failure of commutation of the local 
perspective maps on the event $E$:
\[
    e_i \circ e_j(E) \neq e_j \circ e_i(E)
\]
Finally, privilege is not an ordering relation: it may occur in both directions for the 
same event.
\subsubsection*{The Relativity of Information Frames}
We can now state the relativity of information frames precisely.
\begin{axiom}[The Relativity Of Information Frames]\label{rif}
    After interaction, an event is physically admissible if and only if it does not privilege
    one information frame over another. Equivalently, in the physically admissible joint
    frame, no information frame is privileged over another for any event in its 
    $\sigma$-algebra.
\end{axiom}

More explicitly for any pair of frames $i,j$ of the joint frame and any shared event $E$
we have:
\[
    e_i(E) = e_i(e_j(E))\quad \text{ and }\quad e_j(E) = e_j(e_i(E))
\]
\subsubsection*{The Frame Pointer Algebra}
Next, with \cref{rif} in mind we consider the set of physical events for a given 
information frame $\mathscr I_i$ during an interaction:
\[
    \mathcal F_i^{\text{phys}} := \left\{E \in \iota_i(\mathcal F_{i}) \mid 
    e_{j}\circ e_i(E) = e_j(E) \quad \forall j:\; E \in \iota_j(\mathcal F_j)\right\}
\]
That is, the events that do not privilege $\mathscr I_i$ over any other frame.
\begin{proposition}
    The sets in $\mathcal F_i^{\text{phys}}$ form a $\sigma$-algebra.
\end{proposition}
\begin{proof}
    If $E \in \mathcal F_i^{\text{phys}}$ then, for every $j$:
    \[
        e_j(e_i(E^c)) = e_j(e_i(E)^c) = e_j(e_i(E))^c = e_j(E)^c = e_j(E^c)
    \]
    So $E^c \in \mathcal F_i^{\text{phys}}$.

    And let $E_n$ be a countable collection of RIF-valid events, that is:
    \[
        e_j(e_i(E_n)) = e_j(E_n) \quad \forall n,j
    \]
    Since:
    \[
        e_j\left(\bigcup_n E_n\right) = \bigcup_n e_j(E_n)
    \]
    Given any $j \in I$ we have:
    \[
        e_j\left(e_i\left(\bigcup_n E_n\right)\right) = e_j\left(\bigcup_n e_i(E_n)\right) 
        = \bigcup_n e_j \circ e_i (E_n) = \bigcup_n e_j(E_n) = e_j\left(\bigcup_n E_n\right)
    \]
\end{proof}
We give therefore the special name:
\begin{definition}[Frame Pointer Algebra]\label{def:frame-pointer-algebra}
    The algebra of physically admissible events for a frame is called the
    \textbf{frame pointer algebra}.
    \[
        \mathcal F_i^{\text{ptr}} = \mathcal F_i^\text{phys}
    \]
\end{definition}
\subsubsection*{The Pointer Algebra}
The frame pointer algebras in \cref{def:frame-pointer-algebra} represent the physical
events from the context of each information frame. However, the events that can happen
during a interaction are the events that satisfy rif in general for the contexts
of each interaction.
\begin{definition}[Pointer Algebra]\label{def:pointer-algebra}
    The \textbf{pointer algebra} for the joint frame 
    $\mathcal F_{\mathcal J_I}^* \subseteq F_{\mathcal J_I}$ is the algebra:
    \[
        F_{\mathcal J_I}^* := \sigma\left(\bigcup_{i \in I} \mathcal F_i^{\text{ptr}}\right)
    \]
    \noindent
    When the joint frame is obvious we may simply refer to the pointer algebra as 
    $\mathcal F^*$.
\end{definition}
\subsubsection*{Interpretation}
Local perspective maps do not represent physical operations performed in time, but rather
encode how a given information frame identifies events within a joint description.

A privilege occurs precisely when two frames cannot consistently identify the same event 
without reference to an ordering of perspectives, signaling the absence of a joint 
description for that event.

Frame pointer algebras therefore collect those events that admit a stable interpretation
from the perspective of a given frame during an interaction. The pointer algebra of the
joint frame is generated by all such non-privileging events, and represents the maximal
event structure that can be jointly realized without privileging any information frame.

This algebra captures the emergent classical structure associated with the interaction.
\subsubsection*{Coarse-Graining and the Emergence of Probability.}
\begin{remark}\label{rem:probability-origin}
In this framework, probability is not taken as a primitive notion. Instead, probabilistic 
structure arises as a consequence of coarse-graining enforced by \cref{rif}.   
\end{remark}
 When an
interaction removes distinctions that cannot be jointly maintained across information 
frames, multiple incompatible bookkeeping events are identified as a single physically
admissible event in the pointer algebra. From the perspective of an individual frame,
these identified events are indistinguishable, yet no further structural information remains 
available to discriminate between them.

Any assignment of weights to physically admissible events that is stable under further
coarse-graining and compatible with the structure of the pointer algebra must therefore
take the form of a probability measure. In this sense, probabilities encode the residual
information accessible to a frame after incompatible distinctions have been eliminated,
rather than reflecting intrinsic randomness or ignorance of an underlying reality.
\subsection{Relation to Contextuality}
Structural contextuality makes precise the close relationship between the Relativity of
Information Frames and contextuality in the usual sense. In particular, if a family of
information frames is structurally non-contextual in the sense of \cref{def:struc-context},
then no privileged events arise and the pointer algebra coincides with the full 
$\sigma$-algebra of the joint frame.
\begin{proposition}[Pointer algebra relation to structural contextuality]
    If $\left\{\mathscr I_i\right\}_{i \in I}$ is a family of information frames as and
    $\left\{\iota_i\right\}_{i \in I}$ their corresponding embeddings. Then the following
    hold for the \textbf{pointer algebra of the joint frame} 
    \begin{itemize}
        \item if the family is structurally non-contextual $\mathcal F^*_{\mathcal J_I} = 
            \mathcal F_{\mathcal J_I}$
        \item if the family is structurally contextual $\mathcal F^*_{\mathcal J_I} 
            \subsetneq \mathcal F_{\mathcal J_I}$
    \end{itemize}
\end{proposition}
\begin{proof}
    The proof follows directly from the definitions.
\end{proof}

While this observation already captures a nontrivial physical mechanism - namely, the
elimination of incompatible distinctions - we now establish a stronger and more informative
result. Specifically, we show that the pointer algebra is, in a precise sense, the
\textbf{maximally informative non-contextual algebra} compatible with the given joint 
structure.

Structural contextuality alone is not sufficient for our purposes, as it detects 
contextuality only when one attempts to preserve the full informational structure of
each frame. A similar limitation applies to standard probabilistic contextuality.
However, probability measures provide additional flexibility, allowing one to distinguish
between different $\sigma$-algebras on the joint frame without modifying the underlying
event structure.

We therefore fix a consistent empirical model $\left\{\mathscr I_i\right\}_{i \in I}$ as
in \cref{def:emp-model}. For each context we consider the associated information frame
$\mathscr I_i$, with event algebras representing the support of the corresponding probability
measures, as described previously. Using a choice of embeddings $\{\iota_i\}_{i \in I}$,
we construct the joint frame $\mathcal J_I$.

We begin by defining the following induced measure on the joint frame:
\begin{equation}
    \tilde{\mu_i}(E) := \mu_i(\iota^{-1}_i(E)), \qquad E \in \iota_i(\mathcal F_i)
    \label{eq:glob-meas}
\end{equation}
Which represents the ppushfoward of $\mu_i$ to the embedded event algebra. Restricting this
measure to the frame pointer algebra yields
\begin{equation}
    \tilde{\mu}^{\mathrm{ptr}}_i := \tilde\mu_i \restriction \mathcal F_i^{\mathrm{ptr}}
    \label{eq:res-meas}
\end{equation}
In order to glue these restricted measures into a single global measure on the pointer
algebra $\mathcal F^*$, it is necessary that they agree on overlaps:
\begin{equation}
    \tilde\mu_i^{\mathrm{ptr}}(E) = \tilde\mu_j^{\mathrm{ptr}}(E) \qquad \forall E 
    \in \mathcal F_i^{\mathrm{ptr}} \cap \mathcal F_j^{\mathrm{ptr}}.
    \label{eq:consistency}
\end{equation}
We then state the following lemma:
\begin{lemma}[Pointer Overlap Consistency]\label{lemma:overlap-consistency}
    If $E \in \mathcal F_i^{\mathrm{ptr}} \cap \mathcal F_j^{\mathrm{ptr}}$, then
    $E$ is a \textbf{shared event} whose identification is order-independent, hence
    any empirical family assigns it the same weight.
\end{lemma}
\begin{proof}[Sketch]
    By construction, events in $\mathcal F_i^{\mathrm{ptr}}
    \cap \mathcal F_j^{\mathrm{ptr}}$ admit a frame-independent identification in the joint
    structure. Such events correspond to shared events whose embeddings agree up to
    $\sigma$-algebra homomorphisms, and hence empirical consistency on overlaps implies
    the equality of their assigned weights.
\end{proof}

From this lemma, it is obvious that \cref{eq:consistency} holds, which allows us to state
the desired result
\begin{proposition}[Existence of a non-contextual measure]
    If $\tilde\mu_i^{\mathrm{ptr}}$ agree on overlaps, there exists a probability measure
    $\mu^*$ on $(\Omega_{\mathcal J}, \mathcal F^*)$ extending them. 
\end{proposition}
\begin{proof}[Sketch]
We define a pre-measure $\mu_0$ for $\mathcal F^*$, since the frame pointer algebras are
the generators of $\mathcal F^*$ we can define, whenever $E \in \mathcal F_i^{\mathrm{ptr}}$
\[
    \mu_0(E) = \tilde \mu_i^{\mathrm{ptr}}(E)
\]
Since we have \cref{eq:consistency}, this is well defined. Since 
$\mu_0(\Omega_\mathcal J) = 1$ and $\mu_0 \geq 0$ we can use the 
\emph{Caratheodory Extension Theorem}\cref{Catheodory} to get a measure $\mu^*$ on 
$\mathcal F^*$ with the required overlap consistency.
\end{proof}
This result shows that, starting from a empirical model which may or may not be contextual
on the constructed joint frame, the potential coarse graining to the pointer algebra
directly admits a consistent probability measure with that empirical model.

In what follows, since we had fixed a empirical model and constructed the joint
frame and pointer algebra relative to the $\sigma$-algebraic structure induced by the
that model. Maximality here is understood relative to this joint frame.
\begin{proposition}[Maximality of the pointer algebra]
    Any strict extension $\mathcal F^* \subsetneq \mathcal G \subseteq \mathcal 
    F_{\mathcal J_I}$ fails to admit a consistent global measure extending the fixed
    empirical family.
\end{proposition}
\begin{proof}[Sketch]
    Since $\mathcal G$ must have a event $E$ that is not \textbf{RIF admissible} we have
    for some pair $i,j$:
    \[
        E \in \iota_i(\mathcal F_i) \cap \iota_j(\mathcal F_j) \quad \text{ and }
        \quad e_j(e_i(E)) \neq e_j(E)
    \]
    Since the events are in the support from the initial assumptions and they
    are distinct events we must have:
    \[
        \tilde\mu_j(e_j(e_i(E))) \neq \tilde\mu_j(e_j(E))
    \]
    Then suppose we can build a global measure. 
    But as we have seen, consistency would require 
    \[
        \tilde\mu_j(e_j(e_i(E))) = \tilde\mu_j(e_j(E))
    \]
    So $\mathcal G$ cannot admit a consistent global measure.
\end{proof}
These results imply a strong correlation between contextuality and the RIF pointer algebra.
\begin{theorem}[The Pointer Algebra is the Maximally Informative Non-Contextual Algebra]
    For a joint frame built from a family of information frames the pointer algebra 
    is the largest $\sigma$-algebra that admits consistent probability measures.
\end{theorem}
\subsubsection*{Remarks}
We note that this result is in some sense informal, there is a real limiation on how
contextuality is defined to talk about a maximally informative \textbf{non-contextual} 
algebra, the construction is far more natural in the \textbf{RIF} definition, but the
arguments strongly show the link between the concepts.

It is important to note that if one fixes the joint frame first, and look at what empirical 
families can be built on it. Non-contextuality of the pointer algebra becomes
relative to this joint frame. With this understanding, the pointer algebra
has two key properites:
\begin{itemize}
    \item Any consistent empirical family admits a global probability measure on the 
        pointer algebra.
    \item Any strict extension of the pointer algebra admits a locally consistent 
        probability assignment that do not glue to a global measure.
\end{itemize}
\part{Physical Structure and Consequences}
\section{Quantum Mechanics}
In this section we clarify the connection between the RIF framework and quantum theory.
Our goal is not a full formal reconstruction of quantum meachanics, but to show that
the event structures arising naturally in RIF coincide with the algebraic structures 
underlying quantum theory. In particular, we show that the collection of contextual event
algebras forms an orthomodular lattice, and that standard Hilbert space realizations arise 
under the usual structural assumptions. Consequences such as the Born rule and Wigner-type
consistency conditions will then be seen to follow naturally from RIF invariance.
\subsection{The Hilbert Space Realization}
We will not attempt a full Hilbert space reconstruction in this work. Instead, we situate
the RIF event structure within the well-established framework of quantum logic, highlighting
where RIF provides a natural physical interpretation of the underlying assumptions.
\subsubsection*{The Orthomodular Lattice of Events}
Let $(\mathcal F_i)_{i \in I}$ denote the Boolean $\sigma$-algebras associated to each 
information frame (context), and let the usual embeddings $(\iota_i)_{i \in I}$ into 
a joint frame, as \cref{def:joint-frame},  be given.

From the usual quantum logic perspective, the events on the joint frame algebra form a 
lattice. 
\[
    \mathcal L := \mathcal F_{\mathcal J_I} := \sigma\! \left(\bigcup_{i \in I} 
        \iota_i(\mathcal F_i)\right)
\]
Naturally $\mathcal L$ is a lattice with its operations
\begin{itemize}
    \item complement $E\rightarrow E^c$,
    \item meet: $E \wedge F = E \cap F$,
    \item join: $E \vee F = \text{cl}_\mathcal L (E \cup F)$
\end{itemize}

\begin{proposition}[$\mathcal L$ is a orthomodular lattice]
    The event structure $\mathcal L$ is an orthomodular lattice. Each contextual algebra
    $\iota_i(\mathcal F_i)$ embeds as a maximal Boolean subalgebra of $\mathcal L$.
\end{proposition}
\begin{proof}[Sketch]
    Each $\mathcal F_i$ is a Boolean algebra and therefore an orthocomplemented distributive
    lattice. The embeddings $\iota_i$ preserve complements and finite meets, ensuring
    that $\mathcal L$ is orthocomplemented.

    Distributivity fails in $\mathcal L$ whenever events arising from incompatible frames 
    cannot be jointly refined, which is precisely the manifestation of contextuality in
    the RIF framework. However, the consistency of partial refinements between compatible
    events ensures that the orthomodular law holds. Thus $\mathcal L$ is an orthomodular, but generally 
    non-distributive lattice.
\end{proof}
Under standard assumptions, the same we established in this framework, classical results
in quantum logic \cite{Piron} imply that $\mathcal L$ admits a representation as the
projection lattice of a Hilbert space:
\[\mathcal L \cong \mathsf{Proj}(\mathcal H)\]
where $\mathcal H$ is a Hilbert space over $\mathbb R$, $\mathbb C$ or $\mathbb H$.
\subsubsection*{Relabeling Gauge of the Joint Frame - Unitary Action}
The joint frame construction $\mathcal J_I$ introduces an intrinsic gauge freedom.
Since $\Omega_{\mathcal J_I}$ is a product space over labels $\mathcal C$, different
measurable relabelings of the joint sample space may induce the same relational event
structure. This redundancy is not an additional assumption, but is present by
definition.

Let $\mathrm{Aut}(\mathcal J_I)$ denote the group of measurable bijections
$T:\Omega_{\mathcal J_I}\to\Omega_{\mathcal J_I}$ preserving the joint
$\sigma$-algebra $\mathcal F_{\mathcal J_I}$. Fixing a reference frame $i\in I$, we define
the relabeling gauge group relative to $i$ by
\[
G_i
\;:=\;
\big\{
T\in \mathrm{Aut}(\mathcal J_I)
\;:\;
T^{-1}(\iota_i(\mathcal F_i))=\iota_i(\mathcal F_i)
\big\}.
\]
Elements of $G_i$ correspond to relabelings of the joint description that leave invariant
the event structure accessible to frame $i$, while re-identifying how other contexts are
embedded.

Two joint-frame realizations are said to be gauge-equivalent relative to $i$ when they
are related by an element of $G_i$. The physical content of the RIF construction is thus
identified with structures invariant under this relabeling gauge.

Upon translation to a Hilbert space realization
$\mathcal F_{\mathcal J_I}\cong\mathsf{Proj}(\mathcal H)$, this relabeling gauge is
represented by the familiar projective \textbf{unitary} (and antiunitary) symmetries of the
Hilbert description. Parametrizations of subgroups of $G_i$ may be introduced for
convenience, but are not fundamental at the level of RIF.
\subsubsection*{Complex Structure and Gauge Freedom}

The Hilbert space realization of the joint-frame event structure carries, by construction,
a representation of the intrinsic relabeling gauge of the RIF framework. This gauge is
implemented in the Hilbert description as a projective unitary symmetry acting on
$\mathcal H$.

Supporting a nontrivial and continuous projective unitary action places strong
constraints on the underlying scalar field. Real Hilbert spaces do not admit a
sufficiently rich phase structure to represent generic gauge transformations, while
quaternionic Hilbert spaces introduce additional constraints on the localization and
composition of such symmetries.

By contrast, complex Hilbert spaces provide the minimal setting in which continuous
projective unitary representations, local gauge freedom, and consistent composition of
independent subsystems coexist. From the RIF perspective, the appearance of complex
structure is therefore not an independent postulate, but a natural consequence of
representing the intrinsic joint-frame gauge in a linear space.
\subsubsection*{A Structural Remark on Internal Gauge Symmetry}
The relabeling gauge inherent in the joint-frame construction is represented, upon a
Hilbert-space realization of the framework, as a projective unitary symmetry acting on
$\mathcal H$. At this level, the gauge symmetry is generically very large, reflecting the
freedom in identifying joint descriptions related by relabeling of events.

A further restriction on this relabeling gauge arises from the presence of local
perspective maps and the requirement that no information frame be privileged, as formalized
by \cref{rif}. While $\mathrm{Aut}(\mathcal J_I)$ represents the full descriptive
redundancy of the joint frame, not all such relabelings are compatible with perspectival
symmetry.

In particular, admissible relabelings must preserve the equivalence of descriptions induced
by local perspective maps, in the sense that no information frame can detect a preferred
identification of joint events. This requirement selects a distinguished subgroup of the
relabeling gauge, consisting of transformations that are relationally invisible across all
frames.

Upon Hilbert-space realization, this perspectivally admissible gauge is represented as a
restricted subgroup of the projective unitary symmetry.

We emphasize that no derivation of a specific gauge group is claimed here. Rather, this
discussion is intended to indicate that familiar gauge symmetries, including those of the
Standard Model, are compatible with—and may be viewed as particular reductions of—the
intrinsic relabeling gauge symmetry present in the RIF framework, once restricted to
physically admissible transformations.
\subsection{Observables, Measurements and Operators}
\subsubsection*{Measurement Device}
In the RIF framework, a measurement device is represented by a pure information frame
$\mathscr I_{\mathrm{mes}} := (\Omega_{\mathrm{mes}}, \mathcal F_{\mathrm{mes}})$. The
event algebra $\mathcal F_{\mathrm{mes}}$ encodes the distinctions that the device 
is capable of registering. No additional structure is assumed.
\subsubsection*{Observables}
An observable associated with an information frame $\mathscr I_i$ is a measurable function 
\[
    O_i: \Omega_i \to \mathcal O
\]
Where $\mathcal O$ is an outcome space, such as $\mathbb R$ or a discrete set. Through
the embedding $\iota_i$, each observable induces a corresponding random variable on the
joint frame, defined on the embedded subspace by
\[
    \widetilde{O}_i := O_i \circ \iota_i^{-1} 
\]
The $\sigma$-algebra generated by $\widetilde O_i$ represents the collection of events
distinguishable by the observable $O_i$. Two observables are sait to be \textbf{compatible}
if their induced $\sigma$-algebras generate a jointly Boolean algebra in the joint frame. 
\subsubsection*{Observable-Induced Operators}
To see the relation with operator non-commutativity, it is convenient to consider the
coarse-graining maps induced by observables. Given an event $E$ in the joint frame and an
observable $O_i$, we define the observable-induced map
\[
    e_i^{O_i}(E) := \iota_i\!\left(O_i^{-1}\!\big(O_i(\pi_i(E))\big)\right)
\]
which represents the coarse-graining of $E$ according to the distinctions accessible to
$O_i$. For observables associated with different frames, these maps need not commute:
\[
    e_i^{O_i} \circ e_j^{O_j} \neq e_j^{O_j} \circ e_i^{O_i}
\]
reflecting the incompatibility of the corresponding observables.
\subsubsection*{Measurement}
A measurement of an observable $O_i$ corresponds to a full interaction and conditioning 
the joint description on an event in the frame pointer algebra 
$\mathcal F_i^{\mathrm{ptr}}$. 
Only events belonging to this algebra are physically admissible outcomes of the 
measurement. The coarse-graining implicit in the pointer algebra identifies multiple 
fine-grained events as a single measurement outcome, thereby inducing a 
probabilistic description as per \cref{rem:probability-origin}.

Repeated measurements of the same observable correspond to an already resolved 
conditioning on the same pointer event and therefore yield stable outcomes. In contrast,
an interaction with an incompatible information frame can be seen as a new interaction
that induces a new coarse-graining, reintroducing distinctions that were previously 
suppressed. In this sense, measurement outcomes are not destroyed but rendered 
frame-relative by subsequent incompatible interactions.
\subsubsection*{Relation to POVMs}
When the RIF framework is represented on a Hilbert space, the coarse-grained event 
structure associated with a measurement naturally gives rise to positive operator-valued
measures. Each pointer event $E \in \mathcal F_i^{\mathrm{ptr}}$ corresponds to an
equivalence class of fine-grained events, and the probability assigned to such an event
by the induced global measure may be represented as $\mathrm{Tr}(\rho, E_i)$ for a positive
operator $E_i$.

The non-projective nature of these operators reflects the fact that pointer events are
defined by coarse-graining rather than by sharp partitions of the joint space. In this 
sense, POVMs arise as faithful representations of frame-relative measurements within the
RIF framework. Projective measurements then represent situations where there is no
coarse-graining in the representation.
\subsection{Stern-Gerlach in RIF}
We end with a informal analysis of the Stern-Gerlach test \cite{Stern-Gerlach} using the
RIF framework. We will model the  standard sequence:
\begin{enumerate}
    \item measure spin along $z$ (device A),
    \item measure spin along $x$ (device B),
    \item then measure along $z$ again (device A).
\end{enumerate}
\subsubsection*{Setup}
We begin with modeling the three information frames, the system $\mathscr I_S$, and the
two devices $\mathscr I_A, \mathscr I_B$. After interaction, their frame pointer algebras
are represented by:
\[
    \mathcal F_A^{\mathrm{ptr}} = \sigma(\left\{Z+, Z-\right\}) \quad \text{ and } \quad
    \mathcal F_B^{\mathrm{ptr}} = \sigma(\left\{X+, X-\right\})
\]
\subsubsection*{Interaction 1: System and device A}
The first interaction constructs the joint frame $\mathcal J_{SA}$ using embeddings 
$\iota_S, \iota_A$, then impose RIF and restrict to the pointer algebra $\mathcal F^*_{SA}$.

The outcomes A can see from the measurement is a event in is frame pointer algebra:
\[
    E_A \in \mathcal F_A^{\mathrm{ptr}} \subseteq F^*_{SA}
\]
From A's perspective, repeated measurements in the same interaction context is stable:
\[
    Z+ \text{ then } Z+ \quad \text{or} \quad Z- \text{ then } Z-
\]
Assume for this sequence that the measurement resulted in $Z+$.
\subsubsection*{Interaction 2: System and device B}
Now B interacts with the composite frame $\mathcal J_{SA}$, producing a new joint frame
$\mathcal J_{SAB}$ with its own pointer restriction $\mathcal F^*_{SAB}$. Since the $Z$
and $X$ distinctions are incompatible in the RIF sense, the second interaction induces
a further coarse-graining of the admissible event structure.

The $x$-measurement is not "reading a pre-existing $x$ value"; the interaction is creating
a new coarse-grained joint structure and selecting a event within it. That means, 
conditioning on the earlier $Z+$ event, $B$ sees:
\[
    \mathbb P(X+ \mid Z+) = \mathbb P(X- \mid Z+) = \frac{1}{2}
\]
Again, after the interaction repeated $x$-measurements are stable from $B$'s point of view,
assume that the resulting interaction yielded $X+$.
\subsubsection*{Interaction 3: System and device A again}
Now $A$ is interacting, not with the original $\mathcal J_{SA}$ but with the new composite
that has undergone an incompatible interaction with $B$.

The event labeled $Z+$ after the first interaction does not correspond to the same admissible
event after the incompatible interaction with $B$, since the underlying pointer algebra
has changed
\[
    \mathbb P(Z+ \mid X+) = \mathbb P(Z- \mid X+) = \frac{1}{2}
\]
This does not represent a physical disturbance propagating from $B$ to $A$, but a change
in the admissible joint description induced by an incompatible interaction.
\subsubsection*{Conclusion}
This models the sequences of tests as a series of interaction in RIF. We note that this
analysis is not a final objective truth of what is happening, it is simply a intrepretation
of the mathematics in a easy to reason standard.

Similar intrepretations could be done with a joint frame built by all involved interactions
directly and seeing measurement as a sampling that shifts the events. Or how quantum logic
might normally view such structure.

The main result here, is that interaction is relational, in the RIF framework, a globally
defined event structure is not merely unnecessary but ill-defined as a physical notion.
Any attempt to treat a global description as ontic would privilege descriptions that would 
violate \cref{rif}. Admissible events are therefore defined only relative to interactions,
and no single frame provides a complete description of what occurs.
\subsection{Born Rule Martingale}
In this section we explore how the Born rule appears in this framework. We do not claim,
with the current tools, to recover the quadratic nature of the born rule or its usual 
form. That is left to a reconstruction of the Hilbert space.

\subsubsection*{Setup}
First we consider a particular context's frame $\mathscr I_i$. We define a probability
measure representing that frame $\mu_0$.

In the joint frame we can look at the pushfoward of the contextual measure $\mu_0$ to the
joint $\sigma$-algebra.
\[
\mu := \mu_0 \circ \iota_i^{-1} = \mu_0(\iota_i^{-1}(E)) \quad E \in \mathcal F_\mathcal J
\]
We note that this probability measure does not, nescessarly, represent all contexts of
the joint frame. In fact, in contextual cases, that is not possible. This is simply
the probability of the context $i$ transported to the global frame.
\subsubsection*{The Collapse Filtration}
Since the pointer algebra $\mathcal F^*$ is a sub-$\sigma$ of $\mathcal F_\mathcal J$, we
may consider any decreasing family of $\sigma$-algebras.
\[
    \mathcal F_\mathcal J =: F_0 \supseteq F_1 \supseteq F_2 \supseteq ... 
    \mathcal F^*
\]
representing successive coarse-grainings of the joint description.

While the collapse happens directly to the pointer algebra upon interaction, these 
filtrations represent partial descriptions, this sequence can be interpreted 
heuristically as successive partial information updates - analogous to \textbf{weak or
partial measurements}.
\subsubsection*{The Collapse Martingale}
We can now use $\mu$ as a probability measure to define, for any event $A \in 
\mathcal F_\mathcal J$ we define:
\[
    M_n := \mathbb E_{\mu}\left[ \bold{1}_A \mid \mathcal F_n\right]
\]
As seen \cite{Williams1991} we know this is a Martingale on the reverse filtration, 
and by Doob convergence theorem we have:
\[
    M_n \rightarrow \mathbb E_{\mu}\left[\bold{1}_A \mid F_\text{ptr}\right] \quad 
    \text{(a.s.)}
\]
Thus the conditional probability of $A$ given the pointer algebra coincides with the 
coarse-grained probability $\mu(A)$ in that algebra. Collapse therefore corresponds, in
measure-theoretic terms, to conditioning on the pointer $\sigma$-algebra.
\subsubsection*{Intrepretation}
In this sense, the Born rule emerges as the statement that the observed probabilities are 
those of the conditional measure obtained by coarse-graining to the pointer algebra.

The stochastic character of measurement outcomes is therefore not fundamental but a 
reflection of information loss under coarse-graining.

The martingale formulation expresses the stability of these conditional probabilities under
repeated measurement, as required by empirical repeatability.
\subsection{Wigner's Friend Consistency}
We will now turn our attention to looking at how Wigner's Friend paradox looks like in this
framework.
\subsubsection*{Setup}
For the Wigner friend scenario we acutually have two interactions. First the interaction
of the friend, which we will map to the joint frame $\mathcal J_\text{friend}$. Which
represents the interaction between the system and the friend.

Once that interaction goes through the system has collapsed to the pointer frame
$\mathcal J_\text{friend}^\text{ptr}$. Then Wigner comes and interacts with that joint
frame, forming a new frame $\mathcal J_\text{Wigner}$, which again must collapse to 
$\mathcal J_\text{Wigner}^\text{ptr}$.

We use the fact that, these algebras can all be seen as filtrations of each other to show
that, Wigner cannot assign inconsistant probabilities to the events it can observe.
\subsubsection*{The Sequence of Filtrations}
When the friend interacts with the system, he generates the joint frame 
$\mathcal J_\text{friend}^\text{ptr}$. When Wigner interacts with that frame, a new
joint frame must be built. That joint frame starts by lifting the algebras through the
embeddings $\iota_\text{friend}$ and $\iota_\text{Wigner}$.

The nature of embeddings mean we can consider this all a sequence of filtrations on the
same space. Namely:
\[
    \mathcal F \supseteq \mathcal F_\text{friend}^\text{ptr} 
    \supseteq F_\text{Wigner}^\text{ptr}
\]
With these filtrations in place, we can look at the same style of probability assignments
we had in the born rule.
\subsubsection{The Probabilities}
Now let $A \in \mathcal F_\text{Wigner}^\text{ptr}$
be a event that exists in the final pointe algebra. That is, a event that all systems
involved can talk about.

We can determine the friend's probaility for event $A$ as we did for the born rule 
martingale:
\[
    M_\text{friend} := \mathbb E_\mu \left[\bold{1}_A \mid 
    \mathcal F_\text{friend}^\text{ptr}\right]
\]
Now for Wigner, we have the same:
\[
    M_\text{Wigner} := \mathbb E_\mu \left[\bold{1}_A \mid 
    \mathcal F_\text{Wigner}^\text{ptr}\right]
\]
But since we have
\[
    F_\text{Wigner}^\text{ptr} \subseteq \mathcal F_\text{friend}^\text{ptr} 
    \subseteq \mathcal F
\]
The tower law for martingales gives:
\[
    \mathbb E_\mu \left[M_\text{friend} \mid 
    \mathcal F_\text{Wigner}^\text{ptr}\right] = 
    \mathbb E_\mu \left[\mathbb E_\mu \left[\bold{1}_A \mid 
    \mathcal F_\text{friend}^\text{ptr}\right] \mid 
    \mathcal F_\text{Wigner}^\text{ptr}\right] = \mathbb E_\mu \left[\bold{1}_A \mid 
    \mathcal F_\text{Wigner}^\text{ptr}\right]
\]
Thus, when the friend's description is updated to the Wigner's pointer frame is 
exactly Wigner's own description.
\section{Dynamics}
In this version of the framework, dynamics is not introduced as a primitive law like a 
Hamiltonian flow or differential equation. Instead, dynamics arises from the structure
of interactions between information frames.

Each interaction between contexts induces:
\begin{itemize}
    \item A creation of a joint frame,
    \item followed by collapse to its pointer algebra,
    \item and a pushfoward updated of the preparation measure.
\end{itemize}
Thus, from a fixed frame point of view, the time evolution of it's description of the world
is simply a sequence of coarse-grained $\sigma$-algebras obtained from sucessive interations,
effectively a filtration:
\[
    \mathcal F_\text{frame} =: \mathcal F_0 \supseteq 
    \mathcal F_1 \supseteq \mathcal F_2 \supseteq ... \supseteq \mathcal F_n \supseteq ...
\]
This point of view makes several dynamical properties appear naturally:
\begin{enumerate}
    \item Arrow of Time

        Because each contextual interaction corresponds to additional coarse-graining,
        the evolution of $\sigma$-algebras is monotone (information losing). This
        monotonicity defines a natural direction of time.
    \item Locality Graph

        Interactions occur only between specific frames. The pattern of which frame couples
        to which determines a graph structure, which plays the role of space locality.
    \item Maximum Speed of Influence

        In contextual scenarios, incompatibility forces coarse-graining. The maximal number
        of interaction steps before contextuality appears bounds how "fast" influence can
        propagate along the locality graph.
    \item No Signaling

        Because interactions only merge $\sigma$-algebras along edges of the locality graph,
        and collapse happens locally on the joint frame, no frame can influence another 
        without a mediated interaction.
    \item Gravity

        The coarse grainining imballance %TODO: continues her
\end{enumerate}
None of these dynamical features require any additional axioms. They follow from Axiom~1 
and the definitions of how interactions are built.
\subsection{Arrow of Time}
In this framework, time is not an external parameter. Instead we use the ordering of 
interactions between information frames to induce a canonical direction: each interaction
generates a joint frame where the initial frame algebra is embedded, collapse forces a
coarse graining of its algebra, this monotone loss of distinguishability defines a 
natural arrow of time for that frame.
\subsubsection*{Interaction-induced evolution of $\sigma$-algebras}
To begin, we first fix a information frame $\mathscr I_0$. It then proceeds to interact
with several frames $\mathscr I_{j_1}, \mathscr I_{j_2}$ and so on. 

At each step, a joint frame is first created to host the interaction, Axiom~1 and the 
collapse theorem then converge into a physically realizable joint frame with a sigma algebra
\[
    \mathcal F_{\mathcal J_n}^{\text{ptr}}
\]
That is, the pointer frame on the $n$-th joint frame. Since this is always constructed
by embedding the original $\sigma$-algebra $\mathcal F_0$, we can track its evolution
along the interactions.

We can look at the events of $\mathcal F_0$ that survive in the pointer frame 
$\mathcal F_{\mathcal J_1}^\text{ptr}$ as the $\sigma$-algebra of $\mathscr I_0$ at step
1. That is:
\[
    \mathcal F_1 := \mathcal F_{\mathcal J_1}^\text{ptr} \cap \iota_1(\mathcal F_0)
\]
This gives a natural sequence:
\[
    \mathcal F_0 \supseteq \mathcal F_1\supseteq \mathcal F_1 \supseteq ...
\]
Each step removes events that are incompatible with the new join. The future 
$\sigma$-algebras representations of $\mathcal F_0$ are strictly coarser algebras. There
is no way physical way in the framework to restore the lost distinctions.
\subsection{Locality Graph}
In this theory, \textbf{locality} is not a geometric primitive yet. It is the combinatorial
structure that records which frames have interacted and must therefore share a joint
pointer algebra.
\subsubsection*{The Locality Graph}
Let $\left\{\mathscr I_i\right\}_{i \in I}$ be a family of relevant information frames. We
define the graph:
\[
    G = (V,E)
\]
\[
    (i,j) \in E \quad \iff \quad \text{frames }\mathscr I_i \text{ and } \mathscr I_j
    \text{ have interacted.}
\]
And at every point, an edge represents there is a joint frame where the collapse has
taken place between $(i,j) \in E$.
\subsubsection*{Locality as constraint on the pointer basis}
If frames $i$ and $j$ have interacted, then their local projections $e_i$ and $e_j$ both
act on the same joint frame. Namely the pointer basis there must satisfy:
\[
    \mathcal F_\text{ptr} \subseteq \text{Fix}(e_i)\cap \text{Fix}(e_j)
\]
And if they have not interacted their projetions do not constrain the same $\sigma$-algebra,
which means:
\begin{itemize}
    \item Frames in the same subgraph in $G$ influence one another through constraints in
        the same pointer algebra
    \item Frames in different subgraphs remain completely independent. No constraints 
        propagate between them.
\end{itemize}
This reproduces the operational notion of locality, influence propagates only along paths in
the interaction graph.
\subsubsection*{Dynamics respects the locality graph}
When a new interaction occurs between frames $i$ and $j$ we:
\begin{enumerate}
    \item add an edge $(i,j)$ to $G$;
    \item build the joint frame of all contexts in the connected component containing $i$ 
        and $j$
    \item collapse using all local projections in that component.
\end{enumerate}
Therefore every subgraph of $G$ behaves as local regions, every interaction affects only 
the $\sigma$-algebra generated by its own components.

Locality therefore appears as:
\begin{itemize}
    \item graph locality where edges are interactions
    \item probabilistic locality conditional expectations factor across components of the
        sub graph
\end{itemize}
\subsubsection*{Intrepretation}
This provides a structural notion of spacetime locality:
\begin{itemize}
    \item Information frames as "positions".
    \item Interactions as "lightlike" adjacency.
    \item Paths in G are the only channels through which constraints can propagate.
\end{itemize} 
There is no geometry at this stage, geometry would be additional structure placed on top
of the locality graph.

We emphasize that "locality" and "time" in this theory are entirely structural notions.
They arise from the patterns of interactions between frames, not from any pre-assumed 
geometric spacetime. Any system that can be expressed through interacting information frames
possesses a well-defined locality graph and a induced temporal order, even when no 
traditional spacetime background is specified.
\subsection{Maximum Speed}
In this framework, influence can propagate only through interactions between frames, and
such interactions are encoded by edges in the locality graph $G$.

However contextuality imposes an additional constraint: after a sufficiently long chain
of interactions on mixed degrees of freedom, contextual incompatibility nescessarely appears.
Forcing a coarse-graining of the joint $\sigma$-algebra. This defines a \textbf{time step},
the length of the chain in that time step is always bounded by contextuality.
\subsubsection*{Contextuality forces coarse-graining}
A fundamental structural fact of contextuality (\cite{Abramsky&Brandenburger}) is that
for a sufficiently large family of information frames on shared degrees of freedom, their
combined $\sigma$-algebra on the joint frame cannot remain non-contextual.

While this is not a formal statement, as we could stay replaying interactions on 
non-contextual frames, in realistic scenarios contextuality will show up. At such step,
the pointer algebra will nescessarly coarse grain for some frame. That frame then sees
the length of the chain over the time step as the first interaction step.

Over several interactions, such chains which are always finite, have varying length. But 
any such interaction will have a finite upper bound.
\subsubsection*{Maximum speed for frame $i$}
Once we fix a frame $i$ we see its interaction path in $G$ as:
\[
    i = i_0 \rightarrow i_1 \rightarrow ... \rightarrow i_n = j
\]
In the step $i_n$ is when the first coarse graining happens. We can then define:
\[
    v_i := \sup {n: \mathcal F_n = \mathcal F_0}
\]
That is, the largest number of sequential interactions along which the $\sigma$-algebra of
$i$ remains untouched.
\subsubsection*{Intrepetation}
This pehnomenon should not be interpreted as the emergence of a specific relativistic sped
such as a "speed of light". It is a structural consequence of contextuality: in any 
sufficiently rich system of interacting information frames, repeated interactions inevitably 
produce contextual incompatibilities, which force coarse-graining and thereby prevent
influence from propagating indefinitely without degradation.

The resulting bound is a maximum interaction speed inherent to the information structure
itself. Although different contexts may have different local bounds, a common finite upper
bound always exists for any fixed system.
\subsection{No Signaling}
No signaling is an immediate structural consequence of the locality graph and Axiom~1. 
Because interactions are encoded as edges in the graph, and Axiom~1 applies only on the 
joint frames of the interacting frames, no frame can influence another without a path
of interaction connecting them.
\subsubsection*{Local Independence}
Let $i$ and $j$ represent two frames. If they have never interacted, they lie on different
subgraphs of $G$. Then no local projection $e_i$ of $i$ acts on any algebra observed by $j$.

Thus any mutual time step between then leaves $\mathcal F_j$ untouched, that is:
\[
    \mathcal F_j^{n+1} = \mathcal F_j^n
\]
When that time step operates on $i$, quite simply that is, a frame cannot alter the algebra
of a frame unless they interact.
\subsubsection*{Causal Separation}
If $i$ and $j$ lie in different subgraphs, let $e_k$ bet the local projection of a new
interaction on the subgraph of $i$. Since there is no path from $k$ to $j$ the projection
$e_k$ has no overlap with the algebras $j$ can see. Therefore, $e_k$ has no effect on 
$\mathcal F_j$.

Therefore
\begin{itemize}
    \item Effects of interactions propagate only along paths in the locality graph.
    \item Only frames lying in the same subgraph can influence each other.
\end{itemize}
This is a purely structural notion of causal separation.
\subsubsection*{Intrepretation}
No signaling in this framework is almost tautological, information cannot be transmitted 
between frames that have not interacted.
\subsection{Gravity}
How gravity is the imbalance between algebra %TODO: complete
\subsubsection*{Intrepretation}

\section{Intrepetation and Comparisons}
Here we explore the theory in the context of the broader literature on measurement and
study some of its features, including the implications it has on ontology. We begin with
a informal example of the framework applied to the spin of a electron. To illustrate
how this framework maps to standard Quantum Mechanics.
\subsection{The Ontology of Information Frames}
\subsection{Comparisons - Collapse Models}
\subsection{Comparisons - Intrepetations}
\section{Discussion}
\subsection{Geometry and Dynamics}
\subsection{Reconstruction Programs}
\subsection{Chaos}
\subsection{Implications for gravity}
\section{Conclusion}

\phantomsection
\addcontentsline{toc}{section}{References}
\begin{thebibliography}{9}

\bibitem{Williams1991}
D. Williams,  
\textit{Probability with Martingales},  
Cambridge University Press, 1991.

\bibitem{Pollard2002}
D. Pollard,  
\textit{A User’s Guide to Measure-Theoretic Probability},  
Cambridge University Press, 2002. 
\bibitem{Kallenberg}
O. Kallenberg,  
\textit{Foundations of Modern Probability},  
Springer, 2002.
\bibitem{Csiszar}
I. Csiszár,
\textit{Information-type measures of difference of probability distributions and indirect observation},  
Studia Sci. Math. Hungarica, 1967.

\end{thebibliography}
\end{document}

